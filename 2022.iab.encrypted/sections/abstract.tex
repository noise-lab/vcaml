\abstract{
As users continue to depend on video conferencing applications (VCAs) for remote
participation in work, education, healthcare, and recreation,
ensuring a high quality of experience (QoE) when using VCAs is critical.
Although QoE depends to some degree on the specific circumstances of end
users,
network operators can often play
important role in mitigating QoE degradation resulting from poor local
network conditions.
A network operator who can observe a
VCA's QoE metrics may be able to diagnose and react to QoE degradation,
potentially preventing even transient congestion events from affecting user
experience.
Unfortunately, 
network operators lack direct access to application QoE, and must infer
QoE from the encrypted application traffic as it traverses the network.  

Therefore, we consider the question \textit{if and how we can enable network operators
	to infer QoE
	\footnote{Although QoE is subjective by definition~\cite{hossfeld2016formal}, our 
		focus in this paper is 
		on inferring objective application metrics such as frame rate and frame jitter. Existing 
		work 
		provides methods to model user experience using these application 
		metrics~\cite{banitalebi2015effect}. In addition, although VCA
		performance is determined by both audio and video, in this paper we study only
		video. This is in accordance with previous research in this 
		area~\cite{nikravesh2016qoe,carofiglio2021characterizing}.} 
	for VCAs at per-second time granularity, from passive 
	measurements
	of encrypted network traffic}. Recent work has proposed 
data-driven techniques, often based on machine learning, to estimate VCA QoE metrics from
network-layer
metrics~\cite{yan2017enabling,nikravesh2016qoe,carofiglio2021characterizing}.
Yet, most existing techniques produce coarse-grained inference, such as average frame
rate or mean opinion score of a video session (VCA call).
Therefore, we study whether it is possible to estimate these metrics {\em at a one second 
	timescale}, and ultimately to detect transient QoE degradation events that
would be otherwise undetected in coarser-grained QoE metrics. This fine-grained
visibility enables network operators to perform better post-hoc analysis and network
intervention.

Past work also assumes operators can access and parse
application-level headers which may not be practical in many cases.
Certain VCAs such as Zoom use custom application protocols,
thus making it challenging to extract any information using standard network
monitors~\cite{zoom_rtp}. Moreover, given trends in traffic encryption, 
all application headers may eventually be encrypted~\cite{yang2011security}.
Thus, \textit{we also attempt to estimate video QoE using only IP and UDP
	headers}, and compare the accuracy of the models that use only these headers
to those that also rely on application headers. An additional advantage of this is that existing
network monitoring systems can extract such information at scale~\cite{star-flow}.

With these design goals, we develop a QoE inference algorithm that leverages the
semantics of video delivery in VCA network protocols. A key technical insight
in this research is that due to VCA's real-time nature, each video frame is encoded and its 
packets are sent as a group---sometimes even in a microburst~\cite{salsify}. 
As a result, it is possible to group packets by frame and thus estimate key QoE
metrics at a fine time granularity. We demonstrate this capability using the
application-layers headers of unencrypted traffic as well as using only IP/UDP
headers for encrypted traffic. 


We conduct an in-lab evaluation of our inference technique using data collected from two 
popular VCAs, \meet
and \teams. We assume a 2-person call setting and collect network traces and ground truth 
data for such setting under diverse emulated network conditions. Our evaluation 
demonstrates that we 
can estimate key QoE metrics with
high accuracy. 


\textbf{Future Work}: Our work provides a promising direction to infer fine-grained VCA QoE 
using passive network measurements, that too only IP
and UDP headers. There are key challenges/questions that still remain, including:

\begin{itemize}
	\item Our preliminary evaluation is under controlled settings and over only two VCAs. The 
	inference techniques need to be evaluated under real user setting. This 
	includes consideration of different usage modality (e.g., screen sharing, multi-party call) 
	and multiple VCAs.
	\item Inference techniques need to adapt to changes over time in networks and VCA 
	implementations. Thus, a framework is needed for continual validation and calibration of 
	inference 
	techniques. 
	\item It is important to develop scalable QoE monitoring solutions that can provide QoE 
	estimates  in real time. Such solutions require innovation in developing efficient network 
	monitoring systems as well as developing lightweight inference techniques. 
\end{itemize}
}

